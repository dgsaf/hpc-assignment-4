\documentclass{article}

% - style template
\usepackage{base}

% - title, author, etc.
\title{PHYS4004 - Assignment 4 - Workflow}
\author{Tom Ross - 1834 2884}
\date{\today}

% - headers
\pagestyle{fancy}
\fancyhf{}
\rhead{\theauthor}
\chead{}
\lhead{\thetitle}
\rfoot{\thepage}
\cfoot{}
\lfoot{}

% - document
\begin{document}

The entire code repository can be found at
\url{https://github.com/dgsaf/hpc-assignment-4}.

\tableofcontents

\listoffigures

\clearpage

\section{Interpretation}
\label{sec:interpretation}

The \lstinline{count} process is presented in \autoref{lst:count}.

\lstinputlisting[
label={lst:count}
, caption={
  The process \lstinline{process count} in \lilf{nextflow/main.nf}.
}
, language=
, morekeywords={process, input, output, container, shell, from, into}
, linerange={48-69}
, firstnumber=48
]{../nextflow/main.nf}

\begin{enumerate}[i.]
\item
  \lstinline[otherkeywords={>}]!echo "seed,ncores,nsrc" > results.csv!

  The output of the left hand side (SNR seed, number of cores used and number of
  sources counted) which would normally be sent to \lstinline{stdout} is
  redirected to the destination specified on the right hand side (in this case,
  a file called \lilf{results.csv}) creating/overwriting the file in the
  process.

\item
  \lstinline[otherkeywords={>>}]!echo "${seed},${cores},${nsrc}" >> results.csv!

  Similar to the effect of \lstinline[otherkeywords={>}]!>! above, except that
  this operator appends to the file, rather than overwriting it.

\item
  \lstinline[otherkeywords={|}]!cat ${f} | wc -l!

  The output of the left hand (normally send to \lstinline{stdout}) is
  \textit{piped} to input of the right hand side; that is, the input of
  \lstinline{wc -l} is taken from the output of \lstinline!cat ${f}!.
  This effect of this command is to count the number of lines in the file with
  filepath \lstinline!${f}!.

\item
  \lstinline!$(<command>)! and \lstinline!($(<command>))!

  Wrapping a shell command with a single pair of parentheses,
  \lstinline!(<command>)! executes that command in a sub-shell - prefixing this
  with a dollar sign captures the final output of that subshell
  \lstinline!$(<command>)!.
  However if the result of this is a list of whitespace-delimited strings,
  wrapping this in a further set of parentheses, \lstinline!($(<command>))!,
  will capture the output as an array variable.

\item
  \lstinline!$(ls table*.csv)!

  The effect of \lstinline!ls table*.csv! is to list all files in the current
  directory which start with \lstinline!table! and end with \lstinline!.csv!.
  The effect of \lstinline!$(ls table*.csv)! is to capture this list of matching
  files as a string.
  Specifically, it collects the files containing tables of counted sources for
  all combinations of \lstinline{seed} and \lstinline{cores}.

\item
  \lstinline!echo ${f}!

  The effect of this command is to output (to \lstinline{stdout}) the
  string-value of the variable \lstinline{f}.
  Specifically, this outputs the filepath of a specific file, which is taken
  from the list of tables described above, and will be of the form
  \lstinline!table_<seeds>_<cores>.csv!.

\item
  \lstinline!tr '_.' ' '!

  The effect of this command is, for a given input, to translate all instances
  of the characters \lstinline!_! and \lstinline!.! into whitespace.
  Specifically this transforms \lstinline!table_<seeds>_<cores>.csv! into
  \lstinline!table <seeds> <cores> csv!.

\item
  \lstinline!awk '{print $2 " " $3}'!

  The effect of this command is, for a given input consisting of lines
  (delimited by \lstinline!\n!) containing records (delimited by whitespace), to
  print the second record, followed by a white space, followed by the third
  record, for each line in the input.
  Specifically, this transforms \lstinline!table <seeds> <cores> csv! into
  \lstinline!<seeds> <cores>!.

\item
  \lstinline!cat ${f}!

  The effect of this command is to output the contents of the file, with
  filepath \lstinline!${f}! - specifically
  \lstinline!table_<seeds>_<cores>.csv!.

\item
  \lstinline!wc -l!

  The effect of this command is to, for a given input, count the number of
  newlines (occurences of \lstinline!\n!) in the input.
  Specifically, this counts the number of newlines in the file
  \lstinline!table_<seeds>_<cores>.csv!.

\item
  \lstinline!echo "$(cat ${f} | wc -l)-1" | bc -l!

  The effect of this command is to count the number of sources in the file
  \lstinline!table_<seeds>_<cores>.csv!, by first counting the number of lines
  in the file, and then subtracting for the number of lines which are not source
  data records.

% \item
%   remove this \lstinline!cat ${f}!

\end{enumerate}

\clearpage

\section{Development}
\label{sec:development}

\lstinputlisting[
label={lst:count}
, caption={
  The channel \lstinline{counted_ch} is duplicated, with one for each of
  \lstinline{process plot_for}, and \lstinline{process plot_xargs}, in
  \lilf{nextflow/main.nf}.
}
, language=
, morekeywords={process, input, output, container, shell, from, into}
, linerange={72-72}
, firstnumber=72
]{../nextflow/main.nf}

\subsection{Bash For Loop}
\label{sec:bash-for-loop}

\lstinputlisting[
label={lst:count}
, caption={
  The process \lstinline{process plot_for} in \lilf{nextflow/main.nf}.
}
, language=
, morekeywords={process, input, output, container, shell, from, into}
, linerange={75-93}
, firstnumber=75
]{../nextflow/main.nf}

\subsection{xargs Command}
\label{sec:xargs-command}

\lstinputlisting[
label={lst:count}
, caption={
  The process \lstinline{process plot_xargs} in \lilf{nextflow/main.nf}.
}
, language=
, morekeywords={process, input, output, container, shell, from, into}
, linerange={96-113}
, firstnumber=96
]{../nextflow/main.nf}

\section{Execution}
\label{sec:execution}

\subsection{SNR Plot}
\label{sec:snr-plot}

\begin{figure}[h]
  \centering
  \includegraphics[scale = 0.75]{../nextflow/output/plot_for_1.png}
  \caption[SNR Plot]{
    The Signal-to-Noise Ratio (SNR) is presented for a range of seed SNR values.
    This figure was produced by
    \lstinline[morekeywords={process}]{process plot_for}, for the case of
    \lstinline{cores = 1}.
    No difference was observed between this plot and any of the other plots
    produced for any value of \lstinline{cores}, nor whether if
    \lstinline[morekeywords={process}]{process plot_for} or if
    \lstinline[morekeywords={process}]{process plot_xargs} were used.
  }
  \label{fig:snr-plot}
\end{figure}

\subsection{Workflow DAG}
\label{sec:workflow-dag}

\begin{figure}[h]
  \centering
  \includegraphics[scale = 0.4]{../nextflow/logs/final_dag.png}
  \caption[Workflow DAG]{
    The Directed Acyclic Graph (DAG) of the workflow is presented.
    Note that the \lstinline{combine} operator (below \lstinline{Channel.of})
    appears to have been cropped out by the tool producing the DAG.
  }
  \label{fig:workflow-dag}
\end{figure}

\clearpage

\section{Analysis}
\label{sec:analysis}

\subsection{Resource Usage}
\label{sec:resource-usage}

\subsubsection{CPU Usage}
\label{sec:cpu-usage}

The CPU usage was fairly high across all processes: with a median single-core
CPU usage of 77.6\% for \lstinline{find},  73.7\% for \lstinline{count}, and
90.8\% and 93.5\% for \lstinline{plot_for}, and \lstinline{plot_xargs}
respectively.
Notably the process \lstinline{find} had the widest range of CPU usage, ranging
from a minimum of 61.9\% to 81.0\%.
Overall, these statistics suggest a fairly good single-core CPU utilisation,
although perhaps the processes \lstinline{find} and \lstinline{count} could be
further optimised.

However, the usage of all allocated CPUs, by the processes \lstinline{plot_for}
and \lstinline{plot_xargs}, was 22.7\% and 23.4\% respectively.
It is seen in \autoref{sec:io-usage} that these processes are heavily involved
in I/O operations - which would explain why the single-core CPU usage of these
processes is rather good, while the load-balancing among cores is not.
Many of the cores will be idle while reading and writing operations are
being performed.

\subsubsection{Memory Usage}
\label{sec:memory-usage}

The process \lstinline{find} utilised the largest amount of memory, with a
median memory usage of \SI{224.9}{M}, extending up to a maximum of
\SI{320.9}{M}.
For the other processes, \lstinline{count} had a rather insignificant memory
usage of \SI{4.379}{M}, while \lstinline{plot_for} and \lstinline{plot_xargs}
used \SI{60.3}{M} and \SI{117.5}{M} respectively.
Notably the xargs version of the plotting process was more memory intensive than
the bash for loop version - this reflects that it uses parallel processes to
iterate through the plotting, while the for loop iterates serially.

\subsubsection{Time Usage}
\label{sec:time-usage}

The process \lstinline{find} accounted for the majority of execution time,
averaging from approximately \SIrange{6}{7}{\second} per task for small SNR seed
values to approximately \SI{5}{\second} for larger SNR seed values.
The process \lstinline{count} accounted for a negligble amount of execution
time, requiring only \SI{1.8}{\second}, while the processes
\lstinline{plot_for} and \lstinline{plot_xargs} both required approximately
\SI{4.8}{\second} of execution time.

This is unexpected, as one would imagine the xargs version to be faster, given
that it executed 4 parallel processes, while the bash loop executed the same 4
processes serially.
However, we note that since the I/O usage is the limiting factor for the
\lstinline{plot} processes, specifically reading plain text files from disk
(discussed in \autoref{sec:io-usage}).
Hence, it doesn't that the process are run in parallel, since they will all be
bottlenecked by the read speed - essentially causing the parallel processes to
run serially, as each waits in queue for its data to be read.

\subsubsection{I/O Usage}
\label{sec:io-usage}

The I/O usage was dominated by the processes \lstinline{plot_for} and
\lstinline{plot_xargs} reading from file, with a total of \SI{49.39}{M} being
read for each process.
This is to be expected, as the plotting is performed using data stored in a
plain-text file.
The process \lstinline{count} used a neglible amount of I/O, while
\lstinline{find} read approximately \SI{24.5}{M} per task.

\subsection{Areas for Improvement}
\label{sec:areas-improvement}

The \% usage of all allocated CPUs, by the \lstinline{plot} processes requires
addressing - specifically, to minimise the percentage of time spent performing
I/O.
Currently the script \lilf{plot_completeness.py} is executed to produce a plot
for a given value of \listinline{cores} numerous times - causing numerous reads
of the same file.
This could be improved by modifying the script to instead produce plots for all
cores values in one execution, which would eliminate duplicate file reads, thus
minimising I/O and improving the \% usage of allocated CPUs.


\end{document}
